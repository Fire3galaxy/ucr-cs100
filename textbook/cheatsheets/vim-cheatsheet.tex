\documentclass{article}
\usepackage[utf8]{inputenc}
\usepackage[english]{babel}
\usepackage[T1]{fontenc}
\usepackage[margin=.6in]{geometry}
\usepackage{longtable}
\usepackage{listings}
\usepackage{multicol}
\usepackage{multirow}
\usepackage{color}
\usepackage{tabu}
\usepackage{microtype}

%\makeatletter % To make function descriptions indent when they're long
%\newcommand*\indenth[2]{%
%  \@hangfrom{#1}%
%  {#2}%
%} 
%\makeatother

%Indent w/ static #
\newcommand{\indenth}[1][.5]{\hangindent=#1in
                         \hangafter=1 }

\lstset{ %
basicstyle=\large % For font size
}

\title{CS100 Syscall Cheatsheet}
\author{Read your man pages, and sleep once in a while, k?}
\date{}

\begin{document}
\maketitle

\large This document contains interfaces, libraries, descriptions, and errors for 
syscalls used in CS100. They're organized by category and order of use in the class (as of Winter 2015). For your linux's safety and 
your grade's safety, please use \texttt{perror()} with every syscall. 

\medskip
\begin{tabu} {p{1in}|p{5.7in}}
\texttt{perror()} & \indenth\texttt{void perror(const char* s)} --- prints \texttt{s} and syscall's error message as defined by global int \texttt{errno} to \texttt{stderr}
    \\
& \texttt{stdio.h}, \texttt{errno.h}
\end{tabu}
\normalsize

\section{Syscall interface and libraries}
\begin{longtabu}{p{1in}|p{5.7in}}
    \bf Function & \textbf{Interface, Include, Description}
        \\ \hline
    % fork
    \texttt{fork()} &  \indenth\texttt{pid\_t fork(void)} --- Creates child process. Returns \texttt 0 to child process and child's process id to parent process, or \texttt -1 if error. No child process made if error occurs.
        \\
    & \texttt{unistd.h}
    %
    % exec
        \\ \hline
    \texttt{exec} & \indenth\texttt{Exec} note: If \texttt{exec} succeeds, the current process will end and \texttt{exec} will not return. It returns \texttt{-1} if it fails (e.g. program file not found). \texttt{char *const argv} also must be \texttt{NULL} terminated. \par
        \\
    \texttt{execv()} & \indenth\texttt{int execv(const char* path, char *const argv[])} --- Executes program \texttt path and passes arguments \texttt{argv}. Requires full path name of program.
        \\
    & \texttt{unistd.h}
        \\
    \texttt{execvp()} & \indenth\texttt{int execvp(const char* file, char *const argv[])} Executes program \texttt path and passes arguments \texttt{argv}., finds program file automatically by checking directories in environmental variable \texttt{PATH}
        \\
    & \texttt{unistd.h}
    %
    % wait
        \\ \hline
    \texttt{wait()} & \indenth\texttt{pid\_t wait(int* status)} --- Waits for child process to terminate. \texttt{int* status} stores exit status of child process. Use \texttt NULL if not needed. Returns child pid if succeeds, \texttt -1 if fails (e.g. no child). 
        \\
    & \texttt sys/wait.h
    %
        \\
    \texttt{waitpid()} & \indenth\texttt{pid\_t waitpid(pid\_t pid, int* status, int options)} --- Similar to \texttt wait(). Can specify \texttt pid to wait for specific child; use 0 to wait for any child. Can also wait for stopped processes by adding option \texttt WUNTRACED and check immediately instead of waiting with \texttt WNOHANG (Bitwise OR '\texttt{|}' to combine options). Returns child pid if succeeds, \texttt -1 if fails (e.g. invalid pid).
        \\
    & \texttt sys/wait.h
    % opendir
        \\ \hline
    Directories and Files & \indenth Note: When these functions require a directory or file name, they only require a \itshape relative path\rm. This means that if your process was called in directory \texttt{bin/foo/bar/}, instead of using \texttt{bin/foo/bar/p.cpp}, the \itshape absolute path\rm, as your parameter, you can use \texttt{p.cpp} or \texttt{./p.cpp} instead. \par
        \\
    \texttt{opendir()} & \indenth\texttt{DIR* opendir(const char* name)} --- opens directory stream to directory \texttt{name} and returns pointer to its first entry. Returns \texttt NULL on error. 
        \\
    & \texttt dirent.h
        %
        \\ \hline
    \texttt{closedir()} & \indenth\texttt{int closedir(DIR* dirp)} --- Closes  directory. returns \texttt 0 on success, \texttt -1 on failure
        \\
    & \texttt dirent.h
        %
        \\ \hline
    \texttt{chdir()} & \indenth\texttt{int chdir(const char* path)} --- change directory of calling process to \it path \rm . Returns \texttt 0 on success, \texttt -1 on failure.
        \\
    & \texttt{sys/stat.h}, \texttt{unistd.h}
        %
        \\
    \texttt{stat()} & \indenth\texttt{int stat(const char* path,  struct stat* buf)} --- Gives information about a file in \texttt{struct stat}, e.g. permissions, type of file, time created. See \texttt{ls -l} for example of provided information. Macros are also provided that take \texttt{mode\_t st\_mode} in \texttt{struct stat} and returns true/false e.g. \texttt{S\_ISDIR(st\_mode)}, \texttt{S\_REG(st\_mode)}. Returns \texttt 0 on success, \texttt -1 on failure.
        \\
    & \texttt{sys/types.h}, \texttt{sys/stat.h}, \texttt{unistd.h}
        %
        \\ \hline
    \texttt{open()} & \texttt{int open(const char* pathname, int flags)}
        \\
    & \texttt{int open(const char* pathname, int flags, mod\_t mode)}
        \\
    & \indenth Opens file and returns file descriptor which can be used, with flags, to read/write/create file. Flags: Must use either \texttt O\_RDONLY (read file), \texttt O\_WRONLY (write to file), or \texttt O\_RDWR (both) in call. These can be bitwise \texttt{OR}'d ('\texttt{|}') with other flags. \texttt O\_CREAT creates file if it doesn't exist. \texttt O\_TRUNC overwrites contents of file, O\_APPEND writes at the end of file. When creating files, \texttt mode arguments can be added to specify permissions of new file.
        \\
    & Warning: Every call to \texttt open() must have corresponding \texttt close() or else file descriptors will be left open (similar to memory leaks with \texttt new and \texttt delete).
        \\
    & \texttt{sys/types.h}, \texttt{sys/stat.h}, \texttt{fcntl.h}
        %
        \\
    \texttt{close()} & \indenth\texttt{int close(int fd)} --- Close a file descriptor. Returns \texttt 0 on success, \texttt -1 on failure (e.g. invalid \textt fd).
        \\
    & \texttt unistd.h
        %
        \\ \hline
    \texttt{dup()} & \indenth\texttt{int dup(int oldfd)} --- copies file descriptor \texttt oldfd to the next lowest unused descriptor, returns new file descriptor on success, \texttt -1 on failure. Warning: be sure to close copies after you finish using them. File descriptors are a limited resource.
        \\
    & \texttt unistd.h
        %
        \\
    \texttt{dup2()} & \indenth\texttt{int dup2(int oldfd, int newfd)} --- Copies file descriptor \texttt oldfd in \texttt{newfd}. Closes \texttt newfd if it already exists. Returns new file descriptor on success, \texttt -1 on failure
        \\
    & \texttt unistd.h
        %
        \\
    \texttt{pipe()} & \indenth\texttt{int pipe(int pipefd[2])} --- Returns two file descriptors in \texttt{pipefd} for reading from and writing to. The file descriptors are one way pipes only, hence the need for two. Data written to \texttt pipefd[1] can be read from \texttt pipefd[0]. \texttt pipe is usually associated with piping in bash (e.g. texttt{ls | grep .cpp}). 
        \\
    & \texttt unistd.h
        %
        \\ \hline
    Signals & \texttt SIGINT to interrupt (\texttt{Ctrl+c}), \texttt SIGTSTP to temporarily stop (\texttt{Ctrl+z}), or \texttt SIGSEGV (segfault). \par
        \\
    \texttt{signal()} & \indenth\texttt{sighandler_t signal(int signum, sighandler_t handler)} --- sets signal handler for signal \texttt signum to function \texttt handler. For signal handlers: SIG\_IGN ignores signal and SIG\_DFL uses default handler. You can also use a user-defined handler as an argument. \texttt sighandler\_t is a pointer to a void function that has an int parameter: \texttt{void myhandler(int sig)}
        \\
    & e.g. \texttt{signal(SIGINT, myhandler);}
        \\
    & \texttt signal.h
        %
        \\
    \texttt{sigaction()} & \indenth\texttt{int sigaction(int signum, const struct sigaction *act, struct sigaction *oldact)} --- similar to signal(), but is more portable and conforms to POSIX. Not needed for CS100 projects. Sets signal handler of signal \texttt signum to handler specified in members \texttt sa_handler or \texttt sa_sigaction of \texttt{struct sigaction act} and saves old handler struct to \texttt oldact. You can leave \texttt act or \texttt old \texttt{NULL}.
        \\
    & \texttt
        %
        \\
    \texttt{kill()} & \indenth\texttt{int kill(pid\_t pid, int sig)} --- Sends any signal to process or proccess group. Returns \texttt 0 on success, \texttt -1 on failure.
        \\
    & \texttt{sys/types.h}, \texttt{signal.h}
        %
        \\ \hline
    \texttt{getcwd()} & \indenth\texttt{char* getcwd(char*buf, size\_t size)} --- Returns current working directory on success and copies to \texttt buf of size \texttt size if not \texttt NULL. Returns \texttt NULL on failure.
        \\
    & \texttt unistd.h
        %
        \\
    \texttt{gethostname()} & \indenth\texttt{int gethostname(char* name, size\_t len)} --- Writes hostname (\texttt hammer.cs.ucr.edu) to char array \texttt name with size \texttt len. Returns \texttt 0 on success, \texttt -1 on failure. Note: Char array must be large enough to hold hostname and \texttt NULL char.
        \\
    & \texttt unistd.h
        %
        \\ \hline
    \texttt{} & \indenth\texttt{} --- 
        \\
    & \texttt
\end{longtabu}

\section{Common syscall errors}
\large \rm

\end{document}
