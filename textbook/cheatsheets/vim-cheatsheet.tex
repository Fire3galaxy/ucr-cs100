\documentclass{article}
\usepackage[utf8]{inputenc}
\usepackage[english]{babel}
\usepackage[T1]{fontenc}
\usepackage[margin=.6in]{geometry}
\usepackage{listings}
\usepackage{multicol}
\usepackage{tabu}
\usepackage{longtable}
\usepackage{color}

\lstset{ %
basicstyle=\large % For font size
}
\lstnewenvironment{code}[2]
  {\lstset{label=#1}}
  {}

\title{CS100 Syscall Cheatsheet}
\author{}
\date{}

\begin{document}
\maketitle

\large This document contains interfaces, libraries, descriptions, and errors for 
syscalls used in CS100. I list errors on the back. For your linux's safety and 
your grade's safety, please use perror() with every syscall.

\smallskip
\begin{code}void perror(const char* s)\end{code}
%\code{stdio.h, errno.h} prints argument message \code{s} based on global int %\code{errno}


\medskip
\begin{tabular}{llll}
    \bf Function & \bf Interface & \bf Include & \bf Description\\
    perror() & void perror(const char* s) &  stdio.h &  prints error message\\
    && errno.h & based on global int \it errno \rm
\end{tabular}
\normalsize

\section{Syscall interface and libraries}

pid\_t fork(void) - unistd.h & creates child process. Returns 0 to child process and child's pid to parent process, or -1 if error. No child made if error.
%----------------------------Fork()
%-----------------------------Exec note!!!!!!
\begin{longtabu}{l p{1in} p{4in}}
    \bf Interface & \bf Include & \bf Description
    \\
    \multicolumn{3}{l}{\large\bf Exec\normalsize} 
    \\ \hline
    \multicolumn{3}{l}{If exec succeeds, the current process will end and exec will not return. It returns -1 if it fails (e.g. program file not found). }
    \\
    \multicolumn{3}{l}{char *const argv also must be NULL terminated.}
    \\ \\ \hline
    int execv(const char* path, char *const argv\textbf{[]})& unistd.h & executes programs and passes arguments, requires full path name of program. 
    \\
    & &
    \\ \hline
    int execvp(const char* file, char *const argv\textbf{[]}) & unistd.h & executes programs and passes arguments, finds program file automatically
    \\ \hline
    pid\_t wait(int* status) & sys/types.h & waits for child process to terminate
    \\
%-----------------------------Wait
    & sys/wait.h & %returns pid if succeeds, -1 if fails (e.g. invalid pid)
    \\
    & & %int* status stores exit status of child process. Use NULL if not needed.
    \\ \hline
    pid\_t waitpid(pid\_t pid, int* status, int options) & sys/types.h & similar to wait(), can specify pid, can also wait for stopped processes. Can add option to wait for process to stop (WUNTRACED) or to just check process (WNOHANG). Returns pid if succeeds, -1 if fails (e.g. invalid pid).
    \\
    & sys/wait.h & 
%---------------------------------WAITPID
    \\ \hline
    DIR* opendir(const char* name) & sys/types.h & opens directory stream and returns pointer to its first entry
    \\
    & dirent.h & 
%----------------------------------OPENDIR
% Returns NULL on failure
    \\ \hline
    struct dirent* readdir(DIR* dirp) & dirent.h & means "read directory", returns pointer to struct dirent of next item in directory or null (end of dir/error)
    \\ \hline
    int closedir(DIR* dirp) & sys/types.h & closes  directory
    \\
%---------------------------------CLOSEDIR
    & dirent.h & %returns 0 on success, -1 on failure
    \\  \hline
    int chdir(const char* path) & unistd.h & change directory of calling process to \it path \rm . Returns 0 on success, -1 on failure.
    \\
%----------------------------------CHDIR
    \\ \hline
    int stat(const char* path,  struct stat* buf) & sys/types.h & gives information about a file in struct stat, e.g. permissions, type of file, time created. See \it ls -l \rm for example of provided information. Stat has macros that take st_mode and returns true/false e.g. S_ISDIR(st_mode). Returns 0 on success, -1 on failure
    \\
    & sys/stat.h & 
    \\
    & unistd.h & 
%-------------------------------------STAT
    \\ \hline
    int open(const char* pathname, int flags) int open(const char* pathname, int flags, mod\_t mode) & sys/types.h & opens file and returns file descriptor which can be used, with flags, to read/write/create file. Flags: Must use either O_RDONLY (read file), O_WRONLY (write to file), or O_RDWR (both) in call
\t These can be bitwise OR'd ('|') with other flags. O_CREAT creates file if it doesn't exist
\t O_TRUNC overwrites contents of file, O_APPEND writes at the end of file
\t When creating files, `mode` arguments can be added to specify permissions of new file
Warning: Every call to open() must have corresponding close()
    \\
    & sys/stat.h & 
    \\
    \ \ \ \ \ \ \ \ \ \ & fcntl.h & 
%------------------------------------OPEN
    \\ \hline
    int close(int fd) & unistd.h & Close a file descriptor. Returns 0 on success, -1 on failure (e.g. invalid fd).
%-------------------------------------CLOSE
    \\ \hline
    int dup(int oldfd) & unistd.h & copies file descriptor \it old fd\rm to the next lowest unused descriptor, returns new file descriptor on success, -1 on failure. Warning: be sure to close copies after you finish using them. File descriptors are a limited resource.
    \\ \hline
    int dup2(int oldfd, int newfd) & unistd.h & copies file descriptor \it oldfd\rm in newfd, closes newfd if it already exists, returns new file descriptor on success, -1 on failure
    \\ \pagebreak
    \bf Interface & \bf Include & \bf Description
    \\ \hline
    int pipe(int pipefd\textbf{[}2\textbf{]}) & unistd.h & returns two file descriptors in \it pipefd\rm,
    \\
    & & pipefd\textbf{[}0\textbf{]} $\rightarrow$ read
    \\
    & & pipefd\textbf{[}1\textbf{]} $\rightarrow$ write
%---------------------------------------SIGNAL
    \\ \hline
    sighandler_t signal(int signum, sighandler_t handler) & signal.h & sets signal handler for signal \it signum\rm to signal handler function. Common signals are SIGINT to interrupt (Ctrl+c), SIGTSTP to temporarily stop (Ctrl+z), or SIGSEGV (segfault). For signal handlers: SIG_IGN ignores signal and SIG_DFL uses default handler. You can also use a user-defined handler like void myhandler(int sig). This is what sighandler_t is, a pointer to a void function that has an int parameter: typedef void (*sighandler_t)(int)
%----------------------------------------SIGACTION
    \\ \hline
    int sigaction(int signum, const struct sigaction *act, struct sigaction *oldact) & signal.h & similar to signal(), but is more portable and conforms to POSIX. Not needed for CS100 projects. Sets signal handler of signal \it signum\rm to handler specified in struct sigaction \it act\rm and saves old handler struct to \it oldact\rm 
    \\
%    & & & for sake of space, struct will be listed at end of cheat sheet or omitted altogether. Just know that in struct sigaction, the sa_handler function pointer is default. It is equivalent to the handler for signal(). If sa_sigaction is set to SA_SIGINFO, then the function pointer sa_sigaction is used. It contains the parameter struct siginfo_t, which contains info about the signal, errno, calling pid, etc. The third parameter is unimportant. void (*sa_sigaction)(int, siginfo_t *, void *);
%----------------------------------------PIPE
%returns 0 on success, -1 on failure
    \\ \hline
    int kill(pid\_t pid, int sig) & sys/types.h & sends any signal to process or proccess group
    \\
    & signal.h & 
%----------------------------------------KILL
%returns 0 on success, -1 on failure
    \\ \hline
    char* getcwd(char*buf, size\_t size)& unistd.h & returns current working directory on success and copies to buf of size \it size\rm if present. Returns NULL on failure.
%----------------------------------GETHOSTNAME
    \\ \hline
    int gethostname(char* name, size\_t len) & unistd.h & writes hostname to char array \it name\rm, requires size of array \it len\rm returns 0 on success, -1 on failure. Note: Char array must be large enough to hold hostname and NULL char.
    \\ \hline
    char* getlogin(void) & stdio.h & returns char pointer to current user's username on success, NULL on failure
    \\ \hline
    char* getenv(const char* name) & stdlib.h & returns value of environmental variable \it name\rm or NULL if it doesn't exist
%-----------------------------------GETGRGID
    \\ \hline
    struct group* getgrgid(gid\_t gid) & sys/types.h & returns struct with group information on success, NULL on not finding gid or error (set errno to 0 before calling and check errno to see if error). Requires group id \it gid\rm
    \\
    &grp.h  &
%-----------------------------------GETPWUID
    \\ \hline
    struct passwd* getpwuid(uid\_t uid) & sys/types.h & returns struct with user info related on success, NULL on not finding uid or error (set errno to 0 before calling and check errno to see if error). Requires user id \it uid \rm .
    \\
    & pwd.h &
%-------------------------------------IOCTL
    \\ \hline
    int ioctl(int d, int request, ...) & sys/ioctl.h & sends request to file descriptor d, used to manipulate devices and terminals. Returns 0 on success usually (some devices use return as output value), -1 on error. Request codes will not be listed here. The third parameter is traditionally char* argp.
\end{longtabu}

%\section{Common syscall errors}
%\large \rm

\end{document}
